\documentclass[a4paper, 12pt,oneside]{article} 
%\documentclass[a4paper, 12pt,oneside,draft]{article} 
\usepackage{preamble}
%--------------------- ACTUAL FILE ---------------------- %
\begin{document} 
%%%
	\input{title_page} 
	% Add titlepage
	\clearpage
	\tableofcontents
	\thispagestyle{empty}
	% Add table of contents
	\clearpage
	\pagenumbering{arabic}
	\setcounter{page}{1}
	\section*{Abstract}
	\begin{itemize}
		\item context of the original paper (with their claims) 
		\item our addition/contribution/comment about it
	\end{itemize}
	\section{Introduction}
		Before diving into the analysis, we provide a brief overview of the datasets main features and the models we consider. Our exploratory analysis is available as a Jupyter notebook, and provides more detail. 

		The dataset is composed of throws from 48 people using 44 coins in total. Given the study did not impose strict guidelines for coins to be used, the design is heavily unbalanced. Eighteen coins have only been thrown by a single person, while some of them have been thrown by more than 20 people. Also, more than half the people have only flipped 5 or less different coins, while someone threw 11 different ones. As for the the person-coin pairs, most have fewer or equal to 1000 throws, while some have around 10000 ones. This severe unbalance must be kept in mind during the study, given it can pose challenges during model interpretation. 

		After plotting the same-side rates across people, coin and person-coin combinations, we deemed it relevant to investigate models considering both person and coin as covariates, as well as individual person-coin pairs. Following the advice given on the project statement, we also branched our analysis in Binomial-response GLM and WLS approaches. 

		Motivated by impacts of muscle-memory on the flipping, we additionally investigated aspects such as time-varying same-side rates and memory between successive throws.
	\section{Analysis}
		\subsection{Model Comparison}
			In this section, we introduce and compare different models for the same side success rate. 

			Some GLMs with binomial responses and some WLS ones based on the ... approximation.
			
			Should explain no a priori response transformation ...
			
			For each, the considered formulas in terms of the covariates are:
			\begin{itemize}
				\item \texttt{1}, corresponding to a constant model.
				\item \texttt{1+C(person)}, corresponding to a model with the person as a covariate.
				\item \texttt{1+C(person)+C(coin)}, corresponding to a model with the person and the coin as covariates.
				\item \texttt{1+C(person)+C(coin)+C(person):C(coin)}, corresponding to a model with the person, the coin, and the interaction between the person and the coin as covariates.
			\end{itemize}
			* model 4 could seem redundant due to nesting-main effect, but we ....
			*  

			Should explain why eliminated some covariates ...
			\subsubsection{Tools For Selection}
			\begin{itemize}
				\item when to use AIC vs LRT ?
				\item citing a few things about LRT not miting overfitting 
				\item 
			\end{itemize}
			\subsubsection{WLS Approach}

			\subsubsection{GLM Approach}
			\lipsum[1]
			\begin{table}[htb]
				\centering
				\caption{Model comparison for different models.}
				\label{tab:model-comparison}
				\begin{tabular}{lccc}
				\toprule
				Model & Deviance & AIC & Model DF \\
				\midrule
				\texttt{1} & 3943.48 & 173.97 & 0 \\
				\texttt{1+C(person)} & 3677.51 & 0.00 & 46 \\
				\texttt{1+C(person)+C(coin)} & 3611.12 & 17.61 & 88 \\
				\bottomrule
				\end{tabular}
			\end{table}
			\lipsum[1]
			\begin{figure}[htb]
				\centering
				\includegraphics[width=0.85\textwidth]{GLM_diagnostics.png}
				\caption{Diagnostics for the selected GLM model. (a).}
				\label{fig:glm-diagnostic}
			\end{figure}
			\lipsum[1]
			\begin{table}[htb]
				\centering
				\caption{Likelihood ratio tests between models.}
				\label{tab:llr-comparison}
				\begin{tabular}{llc}
				\toprule
				Tested model & Restricted model & $p$-value \\
				\midrule
				\texttt{1+C(person)} & \texttt{1} & 0.00e+00 \\
				\texttt{1+C(person)+C(coin)} & \texttt{1+C(person)} & 9.61e-03 \\
				\bottomrule
				\end{tabular}
			\end{table}
		\subsection{Unusual Observations}
		\begin{figure}[htb]
			\centering
			\includegraphics[width=0.85\textwidth]{dev_resid_vs_covariates.png}
			\caption{Dev-resid as a function of (a) person and (b) coin.}
			\label{fig:dev-resid-vs-covariates}
		\end{figure}

		\subsection{Zoom on Learning Effects}
		\begin{itemize}
			\item bias comes from start
			\item amount of bias (considerable)
			\item wobble interpretation (consistent with physical model, citing the paper)
		\end{itemize}
		\lipsum[1]
		\begin{figure}[htb]
			\centering
			\includegraphics[width=0.5\textwidth]{learning_effects.png}
			\caption{Learning effects.}
			\label{fig:learning-effects}
		\end{figure}
		\subsection{Memory Effects ??}

	\section{Discussion}
	\section{Conclusion}
	\section*{Acknowledgements}
	\section*{References}
	%\appendix
	%	\section{Runtime Estimation}\label{appendix:runtime_estimation}
%%%
\end{document} 